\section{Mathematical model}
The quadruped robot Robin depicted in \ref{fig:rmoment} has three distinctive parts: the main body, the tail and four actuator-legs. Respectively, the following key coordinate systems should be noticed: $L_w$ - world coordinate system, $L_c$ - geometrical center of construction; $L_B$ - main body's center of mass; $L_T$ - tail's center of mass; and $L_{pi}$ - $i$-th leg point of contact. The proposed mathematical model is derived under the assumption of a stable galloping maneuver which includes vertical (jumping) and horizontal movement of the robot. For such a maneuver, the actuators (i.e. legs) need to develop periodic horizontal and vertical forces at the point of contact with the ground, side force $S_{pi}$ and thrust force $T_{pi}$ respectively. In order to achieve these forces, an adaquate leg joint trajectory needs to be planned. Trajectory planing for the leg joints goes beyonde the scope of this paper, and the reader is addressed to the following papers. 

\begin{figure}
	\centering
	\includegraphics[width=85mm]{./pictures/RobinMoment.pdf}
	\caption{Robin body dynamics}
	\label{fig:rmoment}
\end{figure}

\subsection{Main body dynamics}
In order to simplify the complexity of the analysis, quadruped's main body is modeled as a system of four, equaly displaced cuboid shaped bodies, as shown in Fig \ref{fig:rmoment}. Ideally, when Robin's body is fully symmetric, all four cuboids have the same mass. Since, in most situations, this is not the case, each cuboid has its own mass $m_i$, with its own center $cm_i$ placed in the middle of the cuboid. Each cuboid element center of mass $cm_i$ is then placed at $\vec{r}_i$ distance from the body geometric center, the $L_c$ coordinate system. The overall main body center of mass is then easily calculated as a set of these for cuboid bodies:
 
\begin{equation}\label{eq:CMrobot}
CM=\frac{\sum_{i=1}^{4}m_i\vec{r}_i}{\sum_{i=1}^{4}m_i}
\end{equation}

Next important dynamic parameter is the inertia tensor. Having four cuboid bodys as the basis of the robot model, tensor of inertia can easily be derived by summing all four bodies using the Parallel axis theorem:

\begin{equation}
\tiny
J_R=\sum_{i=1}^{4}\left \{J_{c_i}+m_i\begin{bmatrix}
{_yr_i}^2+{_zr_i}^2 & {_xr_i}\cdot {_yr_i} & {_xr_i}\cdot {_zr_i} \\ 
-{_xr_i}\cdot {_yr_i} & {_xr_i}^2+{_zr_i}^2 & {_yr_i}\cdot {_zr_i}\\ 
-{_xr_i}\cdot {_zr_i} & {_yr_i}\cdot {_zr_i} & {_yr_i}^2+{_xr_i}^2
\end{bmatrix} \right \}
\normalsize
\end{equation}

where, $_xr_i, _yr_i, _zr_i$ represent $\vec{r}_i$ projection in $x,y,z$ respectively, and $J_{c_i}$ is the moment of inertia of the single cuboid element of height $h_i$, width $w_i$ and length $l_i$ as indicated in Fig. \ref{fig:rmoment}:
\begin{equation}
J_{c_i}=\frac{m_i}{3}\begin{bmatrix}
w_i^2+d_i^2 &0&0 \\ 
0 &h_i^2+d_i^2&0\\ 
0 &0& w_i^2+h_i^2
\end{bmatrix}
\end{equation} 

It could be easily shown that if all four cuboids have the same mass and size (i.e. $m,h,d,w$) and are symmetrically displaced from the body center $L_c$, as depicted in Fig \ref{fig:rmoment}, the total body moment of inertia $J_R$ comes down to a symmetric matrix: 
\begin{equation}
J_{c_i}=\frac{m}{12}\begin{bmatrix}
(2w)^2+d^2 &0&0 \\ 
0 &(2h)^2+d^2&0\\ 
0 &0& (2w)^2+(2h)^2
\end{bmatrix}
\end{equation} 
and the body center of mass is placed directly in the center of body coordinate system. For any other arrangement, tensor matrix will not be symmetric, and what is even worse, the body center of mass will be misaligned, which in turn, produces undesired in-air rotations.

\subsection{Leg dynamics}
As it was previously explained, we propose modeling quadruped's legs as active springs. This can be achieved through adequat leg joint path planning, so that when the legs are in the contact with the ground, side forces and thrust forces are produced. The legs are positioned symmetrically around the geometric center, and similarly as quadrotor aircraft exert the following forces and torques on the quadruped main body:

\begin{gather}\label{eq:Forces}
\vec{F_{tot}}=\sum_{i=1}^{4}(\vec{T_{i}}+\vec{S_{i}})\\
\vec{\tau_{tot}}=\sum_{i=1}^{4}\vec{\rho _i}\times(\vec{T_{i}}+\vec{S_{i}})
\end{gather}  

Where $\vec{T_{i}}$ represents the $i$-th cuboid thrust force that causes vertical movement (i.e. hopping), $\vec{S_{i}}$ $i$-th cuboid side force that moves the quadruped in x direction, and $\rho _i$ $i$-th distance between the geometric center and the forces. In reality, one would have the y - component of the side force, but since this component is of the order of magnitude smaller then the x - component, it is not considered in this paper.

For a perfectly symmetric body follows that the total torque (\ref{eq:Forces}) acting on the body is equal to zero, and the total force is the sum of all 4 active springs. For this, the necessary condition is that all four springs have exactly the same parameters (i.e. Force magnitudes, distance from the geometric center, etc.), and thus exert the same forces. In reality, neither the springs are the same, nor there can be a perfectly symmetric body. We write the body torque equation for a non-symmetric body, with the center of mass displaced for $\vec{\Delta CM}$:

\begin{equation}\label{eq:Torques}
\vec{\tau}_{tot}=\begin{bmatrix}
\Delta \textsc{cm}_y \:\bar{T} & -\left ( \Delta \textsc{cm}_x \:\bar{T} +\Delta \textsc{cm}_z \:\bar{S}\right ) & -\Delta \textsc{cm}_y \:\bar{S}
\end{bmatrix}^T
\end{equation}
where $\bar{T}$ and $\bar{S}$ represent the average thrust and side forces of each leg. The displacement of the center of mass is used to model the unbalanced body dynamics. In practice, the displacement vector  $\vec{\Delta CM}$ incorporates all possible imperfections: design symmetry, leg differences, motor dynamics, unballanced forces and so on.

In order to achieve successful gallop locomotion which consists of a jump and forward movement, average forces $\bar{T}$ and $\bar{S}$ have to exist. On the other hand, for a jumping part of the maneuver to remain stable, we need to counteract the undesired torque (\ref{eq:Torques}). Therefore, we propose, adding a tail to the quadruped in order to balance the robot and eliminate the undesired assymetry. 

\subsection{Tail dynamics}
In order to devise a mathematical model of tail dynamics, we propose modeling the tail as a kinematic chain manipulator attached to robot body. Much like in aerial robots, when the robot is in the air, the tail acts freely on its body. We propose using the tail as a dynamic stabilization for the hopping. First we build the kinematic model using Denavit-Hartenberg parameterization method, after that we devise a dynamic model of the arms using recursive Newton-Euler algorithm.
\subsubsection{Complete kinematic and dynamic tail model}
Kinematic chain of the tail is shown in Fig. \ref{fig:rmax}, and the kinematic parameters are shown in table \ref{tab:DHParameters}. Kinematic chain consists of the body coordinate system, 2 tail revolute joints, one virtual prismatic joint in the tail (i.e. tail length parameter), and the tail center of mass coordinate system. As can be seen from DH parameters table, the two revolute joints have no linear, only angular displacement from each other, therefore their masses and moments of inertia are all zero. The third prismatic joint represents the tail length and is modeled as a long stick with infitesimal thickness. Normally, kinematic chains end with a tool at the tip of the last link, but here, the end effector is actually the tail center of mass, and thus it acts as the "tool" coordinate system.

\begin{table}
	\centering
		\begin{tabular}{ccccc}
		\hline
			& $\theta$ & $d$ & $a$ & $\alpha$ \\\hline
			\multicolumn{5}{c}{Body}\\\hline
			$B-0$ & $0$ & $d_B$ & $a_B$ & $0$\\\hline
			\multicolumn{5}{c}{Tail}\\\hline
			Joint 1 & $q_1$ & $0$ & $0$ & $-\frac{\pi}{2}$\\
			Joint 2 & $q_2$ & $0$ & $0$ & $\frac{\pi}{2}$\\
			Virtual joint& $0$ & $q_3$ & $0$ & $0$\\\hline
		\end{tabular}
	\caption{Denavit-Hartenberg Parameters}\label{tab:DHParameters}
\end{table}

The corresponding kinematic chain transformation matrix:
\begin{equation}
T_0^3=\begin{bmatrix}
C_1 C_2 & -S_1 & C_1 S_2 & C_1 S_2 Q_3 \\
C_2 S_1 & C_1 & S_1 S_2 & S_1 S_2 Q_3 \\
-S_2 & 0 & C_2 & C_2 Q_3 \\
 0 & 0 & 0 & 1
\end{bmatrix}
\end{equation}
with a standard abrevation $Cos(q_1):=C_1$ and $Sin(q_2):=S_2$ completes the Kinematic model of the tail. For this analysis we place the tail in the geometric center of the quadruped, $L_C$. Therefore, the distances between $L_0$ and $L_c$, $d_B$ and $a_B$ respectively, are both equal to zero. Using the recursive Newton-Euler method to build upon the kinematic model, we can write the complete dynamic model of the robot, which takes the following general form:

\begin{equation}\label{eq:NEGeneral}
\vec{\boldsymbol{\tau}}_B=\mathbf{D}\left ( \ddot{q_i},\vec{\boldsymbol{\alpha}}_B \right )+\mathbf{C}\left ( \dot{q_i},\vec{\boldsymbol{\omega}}_B \right )+\mathbf{H}\left ( {q_i} \right )
\end{equation}

The first part of \eqref{eq:NEGeneral}, $\mathbf{D}$ takes into account the tail joint accelerations $\ddot{q}_i$ and the body initial angular acceleration $\boldsymbol{\alpha}_B$. Next are the coriolis and centrifugal forces contained in $\mathbf{C}$, where the forces are produced from the coupling of joint speeds $\dot{q}_i$ and body angular speed $\boldsymbol{\omega}_B$. The term $\mathbf{H}$ containes the gravity part of the equation. In a stand-still situation, when the robot is at rest and angular speeds and accelerations,  $\boldsymbol{\omega}_B$ and  $\boldsymbol{\alpha}_B$ are zero vectors, then the accleration, coriolis and gravitational part of the equations take the following forms:
 
\begin{gather}
\tiny
\mathbf{D}\left ( \ddot{q_i},\vec{\boldsymbol{\alpha}}_0 \right )=\frac{1}{24} m_3 Q_3 \begin{bmatrix}
-3C_{q_1}S_{2q_2}\ddot{q_1} -8S_{q_1}\ddot{q_2}\\ -3S_{q_1}S_{2q_2}\ddot{q_1} +8C_{q_1}\ddot{q_2} \\ 8S_{q_2}\ddot{q_1}
\end{bmatrix}^T 
\normalsize
\end{gather}

\begin{figure}
	\centering
	\includegraphics[width=85mm]{./pictures/RobinRepic.pdf}
	\caption{Robin Tail kinematic chain model}
	\label{fig:rmax}
\end{figure}

\subsubsection{Tail inertia tensor}
In Fig. \ref{fig:rmoment}, a tail with its ${L_T}$ coordinate system, placed in its center of mass is depicted. Introducing a new body in the quadruped construction, the overall center of mass will change accordingly:
\begin{equation}\label{eq:CMrobotAndTail}
CM=\frac{\sum_{i=1}^{4}m_i\vec{r}_i+m_T\vec{p}_T}{\sum_{i=1}^{4}m_i+m_T}= \frac{\Delta\vec{\textsc{cm}}+\mu\cdot\textsc{cm}_T}{1+\mu}
\end{equation}
In the previous equation we introduced the mass ratio $\mu=\frac{m_T}{\sum_{i=1}^{4}m_i}$ and $\Delta\vec{\textsc{cm}}$ is the center of mass of the robot, without the tail, from equation (\ref{eq:CMrobot}). If we want to keep the center of mass in the construction center of the robot (i.e. $CM=0$), then the following condition has to be met:
\begin{equation}\label{eq:CMzeroing}
\begin{bmatrix}
\Delta \textsc{cm}_x\\ 
\Delta \textsc{cm}_y\\ 
\Delta \textsc{cm}_z
\end{bmatrix}+
\frac{\mu Q_3}{2}\begin{bmatrix}
C_1S_2\\ 
S_1S_2\\ 
C_2
\end{bmatrix}=0
\end{equation}
that is to say, that the numerator in (\ref{eq:CMrobotAndTail}) has to be zero. Given that in (\ref{eq:CMzeroing}) there exist only two variables, $Q_1$ and $Q_2$ respectively, one can eliminate only two components of the center of mass displacement vector $\Delta \vec{\textsc{cm}}$. Under a reasonable assumption that $\bar{T}\gg \bar{S}$, from equation (\ref{eq:Torques}) follows that for a stable jump one needs to successfully eliminate $x$ and $y$ components of center of mass displacement vector. This assumption is reasonable because in most cases the forward motion is slow compared to the jumping action, and in z direction one needs to overcome the gravity acceleration $g$, making $\bar{T}$ by the order of magnitude larger then $\bar{S}$.

From a strict mathematical point of view, the necessary condition for (\ref{eq:CMzeroing}) to have a solution is:
\begin{equation}\label{eq:CMcondition}
\frac{\mu Q_3}{2}\geq \textup{MAX}\left \{ \left | \Delta \textsc{cm}_x \right |, \left | \Delta \textsc{cm}_y \right |,\left | \Delta \textsc{cm}_z \right |\right \}
\end{equation}
When the condtion (\ref{eq:CMcondition}) is met, the analytical solution to equation (\ref{eq:CMzeroing}) can easily be derived:
\begin{gather}\label{eq:CMsolution}
Q_1=atan2\left({\Delta \textsc{cm}_y},\, {\Delta \textsc{cm}_x}\right)\\
Q_2=asin\left(\frac{2\sqrt{{\Delta \textsc{cm}_x}^2+{\Delta \textsc{cm}_y}^2}}{\mu Q_3} \right)
\end{gather}

