\begin{abstract}
%\boldmath
The main reason to use a tail in quadruped robot locomotion is to apply additional forces/torques to stabilize the motion during flight phases in which the robot has no contact with the ground. This paper presents a Denavit-Hartenberg parameterization based kinematics of the 2 degree of freedom tail together with the Newton-Euler based dynamic model. Impedance based leg control simplifies the leg motion so that it can be modeled as a dampened spring system. Finally, a recursive algorithm that moves the tail in order to balance the robot is proposed. A realistic model of the robot is built in Open dynamic environment and is used to conduct a series of test proving the effectiveness of the proposed algorithm.
%Compared to autonomous ground vehicles, UAVs (unmanned aerial vehicles) have significant mobility advantages and the potential to operate in otherwise unreachable locations. Micro UAVs still suffer from one major drawback: they do not have the necessary payload capabilities to support high performance arms. This paper, however, investigates the key challenges in controlling a mobile manipulating UAV using a commercially available aircraft and a light-weight prototype 3-arm manipulator. Because of the overall instability of rotorcraft, we use a motion capture system to build an efficient autopilot. Our results indicate that we can accurately model and control our prototype system given significant disturbances when both moving the manipulators and interacting with the ground.
\end{abstract}

% IEEEtran.cls defaults to using nonbold math in the Abstract.
% This preserves the distinction between vectors and scalars. However,
% if the conference you are submitting to favors bold math in the abstract,
% then you can use LaTeX's standard command \boldmath at the very start
% of the abstract to achieve this. Many IEEE journals/conferences frown on
% math in the abstract anyway.

% no keywords
