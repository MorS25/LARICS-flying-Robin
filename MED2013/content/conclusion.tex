\section{Conclusion}\label{sec:conclusion}
 
The tail presented in the paper is used as a counterweight, capable of shifting its center of mass so as to balance the body of the robot. To that end, a recursive algorithm that moves the tail in order to balance the robot is proposed. In order to simplify the complexity of the analysis, quadruped's main body is modeled as a system of four, equally displaced cuboid shaped bodies. Varying the mass of each cuboid separately, it is possible to model the imperfections in the robot design. Impedance control implemented in robot leg joints makes it possible to model the complete leg behavior as a spring-mass system.   

Iterative center of mass displacement algorithm is introduced with the purpose of overcoming imperfections of the robot. The algorithm iteratively navigates the tail towards the optimal pose where its center of mass counteracts the shift in the body center. This minimizes the overall center of mass displacement; consequently the generated rotations during hopping become negligible. Four tests are conducted, that show how that algorithm can compensate unwanted rotations after minimum of 7 to 10 consecutive jumps.

Future work includes testing this algorithm on a real robot implementation so as to achieve a completely symmetric construction. Furthermore, dynamic stabilization based on the conservation of momentum should be implemented. Using these two methods should yield a stable quadruped dyanarobin galloping motion.
