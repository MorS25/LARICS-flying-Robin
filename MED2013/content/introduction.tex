\section{Introduction}\label{sec:introduction}
% no \IEEEPARstart

One of the chalanging tasks in biologically inspired robotics is to use tail to help robot locomotion in the same way as animal do. Ballance control while running and hopping, flight control or climbing are some examples how tail is helping animals to provide robust locomotion. The main reason to use tail in quadruped robot locomotion is to apply additional force/moments to stabilize motion gaits during flight phase when robot has minimal contact with the ground. 

Uniroo \cite{zeglin1991uniroo} is one of the first robot which uses single degree of freedom(DOF) active tail to maintain constant body pitch while emulate hopping kangaroo. A lizard-inspired active tail stabilization presented in \cite{conf/iros/Chang-SiuLTF11} uses tail on wheeled car which provides robust drive over difficult terrain. The momentum based approach is used only for pitch control. The MIT Cheetah robot \cite{DBLP:conf/iros/BriggsLHK12} uses tail to improve running performance up to 30mph. The authors analize also several others artificial mechanisms which can be used to apply non-contact forces/moments like reaction wheel, reaction mass, thrusters etc. 

This paper's goal is to use tail to improove locomotion on quadrupedal hooping robot. The latest research in robot leg design emphasize the use of springs as mechanical part which can accumulate energy in the same way as muscles, tendons and ligaments do in nature world\cite{Zeglin_1999_3268}\cite{Geyer06compliantleg}\cite{Hutter}\cite{RunningSpringer}. At the same time variable stiffnes allows robot to run over a large variety of terrains while adjusting their leg stiffness\cite{Galloway}\cite{Jun:2009:DSV:1703775.1704089}\cite{Hurst_2004_4785}. All this research suggests that quadrupedal robot can be dynamically modeled as mass supported on four spring legs as shown in Fig 1. In order to provide stable hopping, additional tail must be included which can statically affect to mass redistribution and dynamically to compensate unwanted roll and pitch rotation. 


